\documentclass[footmark=none]{tubaf-thesis}
\KOMAoptions{%
    listof=totoc,
}
\usepackage[sans]{tubaf-fonts} % Arial
%\usepackage[]{tubaf-fonts}    % Times
\usepackage{graphicx}
\usepackage{booktabs}
\usepackage[ngerman]{babel}
\usepackage[style=ieee, backend=biber, sorting=none, doi=false, isbn=false, url=false]{biblatex}

\usepackage{hyperref}

\subject{Seminararbeit}
\title{Energieeffiziente Eingebettete Systeme:\\Dynamische Spannungs- und Frequenzskalierung}
\author{Ben Weckend\\{\small Matrikel: 67551}}
\date{30. August 2024}
\examiners{Prof. Dr. Bernhard Jung \and M.Sc. Robert Lösch}
\course{Diplom Robotik}


% Document
\begin{document}

    \pagenumbering{roman}

    \maketitle
    \makedeclarationofauthorship[30. August 2024]
    \tableofcontents
    
    \newpage
    \listoffigures 						% Abbildungsverzeichnis
    \newpage
    \listoftables 						% Tabellenverzeichnis
    
    \clearpage
    \pagenumbering{arabic}
    
    \chapter{Einleitung}
    \cite{5545490}
    	\section{Hintergrund und Motivation}
   			
    	\section{Zielsetzung der Arbeit}
    
    	\section{Struktur der Arbeit}
    
    
 
    \chapter{Grundlagen}
    
    	\section{Eingebettete Systeme und deren Anwendungen}
    
    	\section{Energieeffizienz in eingebetteten Systemen}
    
    		\subsection{ASIC Design}
    
    	\section{Dynamische Spannungs- und Frequenzskalierung Konzepte und Prinzipien}
    
    
    
    \chapter{Techniken zur Dynamischen Spannungs- und Frequenzskalierung}
    
    	\section{Frequenzskalierung}
    	
    		\subsection{Dynamische Frequenzskalierung}
    		
    		\subsection{Thermisch bedingte Frequenzskalierung}
    
    	\section{Spannungsskalierung}
    	
        	\subsection{Statistische Spannungsskalierung}
        	
        	\subsection{Adaptive Spannungsskalierungstechniken}
        	
		\section{Aufbau und Funktion eines Halbleiterbausteins}
        
        	\subsection{Spannungsskalierung und Transistoren}
        	
        \section{Kombinierte Spannungs- und Frequenzskalierungstechniken}
        
        
        
	\chapter{Implementierung von DVFS in eingebettete Systeme}
       
       	\section{Hardwareanforderungen für DVFS}
       	
       	
       	
	\chapter{Grundlegenden Stromspar-Mechanismen}
	
		\subsection{Gating}
			
			\subsubsection{Power Gating}
		
			\subsubsection{Clock Gating}
		
		\subsection{Cache-Hierarchien optimieren}
		
		\subsection{Sleepmodus}
		
		\subsection{Standby-Modus}
		
		\subsection{Power-Down-Modus}
       
       		
       	
       	
       	
	\chapter{Anwendungen und Fallstudien}
       	
		\section{DVFS in mobilen Geräten und Wearables}
        	
        \section{DVFS in IoT-Geräten und Sensornetzwerken}
        	
        \section{DVFS in Weltraumtechnik und Sonden}
        	
        	
        	
        \chapter{Herausforderungen und Zukunftsaussichten}
        
        
        
	\chapter{Schlussfolgerung}
        
        \section{Zusammenfassung der wichtigsten Ergebnisse}
    
    	\section{Bewertung der Rechercheergebnisse}
    		
    	\section{Ausblick auf zukünftige Forschungsrichtungen}
    
    
    
   
    



	\bibliographystyle{plain}
	\bibliography{trueRef}

%    \nocite{*} % include all entries in the bibliography for testing
    %\printbibliography[heading=bibintoc,title=Literaturverzeichnis]

\end{document}
